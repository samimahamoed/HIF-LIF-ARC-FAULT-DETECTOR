The module is responsible for external serial communications.

Here, two interface helper functions are provided for data transmission and reading incoming message (see \hyperlink{a00030_a96a3f016ca5b0736424c2695fe9fbdf8}{serial\+\_\+send\+\_\+response()}, serial\+\_\+get\+\_\+command(,,)).

In these application only algorithim module can access these functions. Basically the implemented operation principle is that, during receiver interrupt the module collects incoming data to receive buffer and sends interrupt notification to algorithm module after successful validation procedure (see \+\_\+\+U1\+R\+X\+Interrupt()). On the other hand, for data transmission request the module copies the message to transmit buffer and starts sending out data on every transmit interrupt (see \+\_\+\+U1\+T\+X\+Interrupt()). Here, sending can be configured in blocking mode or non-\/blocking mode. In the latter case data which can?t fit into the buffer will be lost (see serial\+\_\+send(,)). Whereas, in the first case (the default) the serial message will be blocking till the transmission is completed. However, this doesn?t affect protection functionality serviced by high priority interrupt. 