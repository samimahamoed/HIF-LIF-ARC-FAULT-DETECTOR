\hypertarget{index_IDE}{}\section{I\+D\+E}\label{index_IDE}
The project is created with M\+P\+L\+A\+B 8.\+92 , However, The new M\+P\+L\+A\+B X can also be used to open the project\hypertarget{index_files}{}\section{File organisation}\label{index_files}
The project files have been placed in following directories
\begin{DoxyItemize}
\item mplab\textbackslash{}doxygen -\/ Doxygen related configuration files and doxygen output
\item mplab\textbackslash{}includes -\/ Public include files from all functional/service/system modules
\item mplab\textbackslash{}includes\+\_\+prv -\/ Private include files from all functional/service/system modules. These files are private for modules.
\item mplab\textbackslash{}services -\/ Low-\/level stateless driver modules ( called services in this project)
\item mplab\textbackslash{}system -\/ O\+S files and Microchip processor specific files
\item mplab\textbackslash{}modules -\/ O\+S dependent and product specific system modules
\end{DoxyItemize}\hypertarget{index_os}{}\section{Operating system}\label{index_os}
The O\+S used in this project is simple event based scheduler. See here for details\+: \hyperlink{a00012}{Operating system design}\hypertarget{index_measurements}{}\section{Code timing}\label{index_measurements}
S\+W 18.\+05.\+2014 The main load comes from 100us interrupt which scans sensors. C\+P\+U load
\begin{DoxyItemize}
\item 80\% of C\+P\+U time from the fast timer interrupt
\item 30\% of C\+P\+U time is used of operating system routine with the exception 2.\+5s during startup
\end{DoxyItemize}\hypertarget{index_genpri}{}\section{Generic principles}\label{index_genpri}

\begin{DoxyItemize}
\item The O\+S is started in ( \hyperlink{a00048}{main.\+c} ) file together with some O\+S independent services
\item The O\+S calls ( by sending system notification) to all modules ( see e.\+g. \hyperlink{a00021_a93269cdec3e21934aa9395440a2de605}{algorithm\+\_\+notifyx()} and algorithm\+\_\+init() , all modules are desgined like that )
\item The modules prepare system objects ( event/notification objects for using during runtime) and schedule timers for its periodical operations ( e.\+g. analog\+\_\+init() )
\item The Algortihm module ( \hyperlink{a00038}{algorithm.\+c} ) launches N\+O\+V read and sends configuration notifications to other modules
\item The modules execute supervison timers, in case of any alert send notifications to Algorithm module.
\item The Algorithm module gets all notifications and sends event to itself to process them one by one in its main task loop. See \hyperlink{a00021_a93269cdec3e21934aa9395440a2de605}{algorithm\+\_\+notifyx()} and \hyperlink{a00016_a42ed16c7ef20e0c0031fe7ba7ae377b3}{algorithm\+\_\+wake()}
\end{DoxyItemize}\hypertarget{index_wdglist}{}\section{Watchdog related processing}\label{index_wdglist}
See \hyperlink{a00010}{Watchdog initialization.}\hypertarget{index_nov}{}\section{Non volatile storage}\label{index_nov}
See \hyperlink{a00006}{Non volatile storage}\hypertarget{index_hwio}{}\section{Software validation}\label{index_hwio}
The firmware supports hardware reading redirection to allow testing firmware functionality. Basically, user can replace real hw reading with values supplied by serial commands\hypertarget{index_requirements}{}\section{List of requirements supported directly by the code}\label{index_requirements}
See req 